\documentclass[notitlepage,11pt]{article}
\usepackage[margin=1in]{geometry}
\usepackage{enumitem}
\usepackage{textcomp}
\usepackage{amsmath}
\usepackage{listings}
\usepackage{color}

\definecolor{codegreen}{rgb}{0,0.6,0}
\definecolor{codegray}{rgb}{0.5,0.5,0.5}
\definecolor{codepurple}{rgb}{0.58,0,0.82}
\definecolor{backcolour}{rgb}{0.95,0.95,0.92}
 
\lstdefinestyle{mystyle}{
    backgroundcolor=\color{backcolour},   
    commentstyle=\color{codegreen},
    keywordstyle=\color{magenta},
    numberstyle=\tiny\color{codegray},
    stringstyle=\color{codepurple},
    basicstyle=\footnotesize,
    breakatwhitespace=false,         
    breaklines=true,                 
    captionpos=b,                    
    keepspaces=true,                 
    numbers=left,                    
    numbersep=5pt,                  
    showspaces=false,                
    showstringspaces=false,
    showtabs=false,                  
    tabsize=2
}

\lstset{style=mystyle}

\linespread{1.5}
\setlength{\parindent}{1cm} % Default is 15pt.

\title{Capstone Project Report}
\author{Robbie Merillat : RFID Scanning and User Managment}
\author{Robert Schreibman : Game Integration and Housing Construction}
\date{\today}

\begin{document}
    \maketitle
    \begin{abstract}
        In order to speed up the process of logging into an account with a 
        username and password, this project uses an RFID reader which is 
        used to scan a wallet-sized RFID card in order to automatically log 
        into a gaming account. The gaming data and account information is then 
        centrally stored in a file where the associated user can access it either 
        via RFID card tag or using a regular username and password.
    \end{abstract}

    \newpage

    \section{Project Description}
        The broad idea behind this project is to speed up a process of logging into 
        an account. In today’s society, just about every website we visit has required 
        an account set up with a unique username and password. After setting up dozens of 
        these accounts, each with different info, it becomes impossible to remember each 
        specific username and password for every individual website. For convenience, 
        many individuals have allotted this task to third party password managers to 
        keep track of this information. However, for those who do not trust third 
        party password managers, we have created an alternative system that will do 
        the same job. Using an RFID reader, we have interfaced wallet-sized RFID 
        cards with account login information. 

        Currently, our system is interfaced with a single account associated with two terminal games. After a user scans their RFID card, they will be automatically logged into their account where their game statistics are kept per game. The user then can choose one of two games: tic tac toe or four in a row. After playing their game statistics will be saved to their RFID account.
    \section{Sensors}

    \section{Obstacles}

    \section{Results}

    \section{Summary}

    \section{Code}
        \textbf{"run.py begins our code and starts the RFID reader"}
        \lstinputlisting[language=Python]{../run.py}
\end{document}
